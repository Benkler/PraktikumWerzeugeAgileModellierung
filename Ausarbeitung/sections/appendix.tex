%% LaTeX2e class for student theses
%% sections/apendix.tex
%% 
%% Karlsruhe Institute of Technology
%% Institute for Program Structures and Data Organization
%% Chair for Software Design and Quality (SDQ)
%%
%% Dr.-Ing. Erik Burger
%% burger@kit.edu
%%
%% Version 1.3.3, 2018-04-17

\iflanguage{english}
{\chapter{Appendix}}    % english style
{\chapter{Anhang}}      % german style
\label{ch:appendix}


%% -------------------
%% | Example content |
%% -------------------
\section{Kofigurations-Dateien }
\label{sec:appendix:BPMN Models}
Die folgenden Konfigurationen sind für die Simulation benutzt worden. Zu finden sind diese auch im Git-Hub Repository \cite{github}. Dort befindet sich auch die \textit{workload.json}, die aus Platzgründen nicht in dieses Dokument aufgenommen wird.

\begin{lstlisting}[language=json,firstnumber=1, caption={clock.json}]
{
  "millisecondsTillPublishInfrastructureState": 1000,
  "millisecondsTillPublishQueueState": 1000,
  "intervalDurationInMilliSeconds": 500,
  "millisecondsTillWorkloadChange": 1000,
  "experimentDurationInMinutes": 7
}
\end{lstlisting}

\begin{lstlisting}[language=json,firstnumber=1, caption={infrastructure.json}]
{
  "virtualMachineType": 
    {
    "millisecondsPerTask": 278,
    "vmStartUpTimeInMilliSeconds": 4500
    },
  "amountOfVmsAtSimulationStart": 1,
  "cpuUitilizationWindow" : 10
}
\end{lstlisting}

\begin{lstlisting}[language=json,firstnumber=1, caption={queue.json}]
{
"queueLengthMax": 80000,
"windowSize" : 20,
"queuingDelayInMilliSeconds": 0 
} 
\end{lstlisting}


\begin{lstlisting}[language=json,firstnumber=1, caption={autoscaler.json}]
{
  "lowerThreshold": 0.25,
  "upperThreshold": 0.75,
  "vmMax": 20,
  "vmMin": 1,
  "timeInMsTillNextScalingDecision": 1000,
  "cpuUtilWindow" : 10,
  "queueLengthWindow": 1,
  "coolDownTimeInMilliSeconds": 10000
}
\end{lstlisting}













